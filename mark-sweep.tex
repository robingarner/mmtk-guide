\chapter{Mark-Sweep in MMTk}

\label{chap:mark-sweep}

\mmtk's implementation of mark-sweep is a highly optimized state-of-the-art
collector, and until recently its generational counterpart, GenMS, was the
default production collector.  The mark-sweep policy that underlies the
plan is used in every \mmtk\ collector since it is used for the non-moving and
code spaces common to all plans.

The basic idea is shared with McCarthey's 1960 collector~\citep{McCarthy:60}:
that the collector performs a transitive closure over the object graph marking 
live objects, and that all objects not marked are considered dead
at the end of the collection cycle.  We traditionally refer to marked objects as
being coloured black, and unmarked objects as white.
\begin{figure}
\begin{lstlisting}
/* 
 * Initially, all objects in the heap are coloured white.
 * white, gray and black are disjoint sets of objects
 */              
for (ref : root-set)
  if (ref != zero)
    gray.add(ref.load())

while (!gray.isEmpty())
  obj = gray.get();
  for (ref : obj.nonNullReferences())
    child = ref.load()
    if (child.isWhite())
      gray.add(child)
  black.add(obj)
/*
 * Now all objects are either black or white, white ones
 * are unreachable and can be freed.
 */
for (obj : white)
  obj.free()
\end{lstlisting}
\caption{Mark Sweep: pseudo-code description}
\label{fig:ms:pseudo}
\end{figure}
The pseudo-code in Figure~\ref{fig:ms:pseudo} illustrates the algorithm in terms
of the traditional tri-colour abstraction from \cite{DLM+:76}.  

As with all \mmtk\ plans, collection is performed in parallel, with a number of
collector threads running in parallel to mark the heap.

The abstract description in Figure~\ref{fig:ms:pseudo} requires that each object
can be assigned to one of three sets: white, black and gray.  Traditionally this
is implemented using a boolean flag per object (either stored in the object or
outside the heap).  If the flag is false, the object is white.  If the flag is
true, then either the object is gray or black.  The set of gray objects is
usually kept as a side data structure such as a stack or a
queue, since the purpose of the 'grey' state is as a list of work
still to be done\footnote{There are other more arcane implementations that can
thread the grey set through the heap.  This is generally only of interest in 
space-constrained LISP-like languages where most objects are only 2 words long.}.

\mmtk\ supports two implementations of the mark state, described in
Section~\ref{sec:ms:state}.  The mark-state is either stored in the header
of each object, or in a region reserved at the start of every 4MB chunk of
virtual address space.

One of the shortcomings of pure mark-sweep collectors is that they must traverse
the heap twice in each collection cycle, once to mark the reachable objects
(the mark phase), and then a second time to build the free list from the white
objects (the sweep phase).  
Black objects can also be marked white during the sweep phase, 
to initialize them ready for the next GC.
The sweep phase is particularly expensive because it requires visiting the
entire heap, not just the (hopefully much smaller) fraction that is in use.
\mmtk\ uses \emph{lazy sweeping} \citep{Hughes:82, Boehm:00} to ameliorate the
cost of sweeping.  The \mmtk\ implementation is described in detail below in
Section~\ref{sec:ms:lazy}.

\section{Segregated Free List}

For allocation, the mark-sweep policy uses a \emph{segregated free list}
allocator, described in detail in Section~\ref{sec:alloc:free-list}.  The
essential features relevant here are that a) the heap is divided into
\emph{cells} of ~20 pre-defined \emph{size classes}.  If there are no free cells
of a given size class, a \emph{block} of 4-64KB is allocated and divided into
cells of the given size class.  Within a block, all cells are therefore of the
same size class, so as long as we know the size class of each
block, it is possible to locate each cell (and therefore each object) in the block.

The collector maintains a list per size-class of all blocks containing live
objects.  Each mutator maintains a list per size-class of the blocks it has
allocated into, and a pointer to the first free cell of each size class.

\begin{figure}
\begin{center}

\begin{tikzpicture}



\newcommand*{\heads}[1]{
\draw (0,0) node [draw,shape=rectangle,minimum width=10cm,minimum height=1cm,
                      anchor=south west]
            (#1) {$\cdots$};

 \draw (0.1,0.1) node [draw,shape=rectangle,minimum width=8mm,minimum
height=8mm, anchor=south west] (#1-1) {}; 
\draw (1.1,0.1) node [draw,shape=rectangle,minimum width=8mm,minimum height=8mm,
                      anchor=south west] {}; 

\draw (8.1,0.1) node [draw,shape=rectangle,minimum width=8mm,minimum height=8mm,
                      anchor=south west] {}; 
\draw (9.1,0.1) node [draw,shape=rectangle,minimum width=8mm,minimum height=8mm,
                      anchor=south west] {}; 
}

\tikzstyle {block}=[draw,shape=rectangle,
     minimum width=20mm,minimum height=40mm, anchor=south west];

\tikzstyle {cell}=[draw,shape=rectangle,
     minimum width=18mm,minimum height=8mm, anchor=south west];

\tikzstyle {object}=[cell, fill=gray!25];


\begin{scope}[yshift=9.6cm]
\heads{bps}
\draw (bps.north) node [above] {Block Lists};
\end{scope}
\begin{scope}[yshift=1cm]
\heads{cps}
\draw (cps.north) node [above] {Free Cell Lists};
\end{scope}



\begin{scope}[yshift=3cm]
%\draw (0,0)  -- ++(0,4) -- ++(2,0) -- ++(0,-4) -- ++(-2,0);
\draw (0,0) node [block] (b1) {};
\draw (0.1,0.1) node [cell] (o14) {};
\draw (0.1,1.0) node [cell] (o13) {};
\draw (0.1,1.9) node [cell] (o12) {};
\draw (0.1,2.8) node [object] (o11) {};
\end{scope}

\begin{scope}[yshift=3cm, xshift=3cm]
%\draw (0,0)  -- ++(0,4) -- ++(2,0) -- ++(0,-4) -- ++(-2,0);
\draw (0,0) node [block] (b2) {};
\draw (0.1,0.1) node [object] (o24) {};
\draw (0.1,1.0) node [object] (o23) {};
\draw (0.1,1.9) node [object] (o22) {};
\draw (0.1,2.8) node [object] (o21) {};
\end{scope}

% List of blocks
\draw [->,thick] (bps-1.center) .. controls +(left:2cm) and +(left:1cm) .. 
           node (bp1) {}
           (b1.north west);

\draw [->,thick] (b1.north)+(0,-2mm) .. controls +(up:5mm) and +(up:5mm) ..
           node (bp2) {}
           (b2.north west);

% List of cells
\draw [->] (cps-1.center) .. controls +(left:2cm) and +(left:2cm) .. 
           node (cp1) {}
           (o12.north west);

\draw [->] (o12.center) .. controls +(left:2cm) and +(left:1cm) ..  
           node (cp2) {}
           (o13.north west); 

\draw [->] (o13.center) .. controls +(left:2cm) and +(left:1cm) .. 
           node (cp3) {}
           (o14.north west); 

% Labels of free list pointers
\draw (-3,5) node [minimum width=20mm,align=left] (fltext)
  { List of free cells\\in size class 1 };
  
\draw [->,dotted] (fltext.east) -- (cp1);
\draw [->,dotted] (fltext.east) -- (cp2);
\draw [->,dotted] (fltext.east) -- (cp3);

% Labels of block list pointers
\draw (2,8.5) node [minimum width=20mm,align=left] (bltext)
  { List of blocks\\in size class 1 };
  
\draw [->,dotted] (bltext.west) -- (bp1);
\draw [->,dotted] (bltext) -- (bp2);
\end{tikzpicture} 
\end{center}
\caption{Blocks and cells in the segregated free list}
\end{figure}

\section{Lazy Sweeping}
\label{sec:ms:lazy}



\section{Mark-state implementation}
\label{sec:ms:state}




\begin{lstlisting}[name=Mark Phase, 
                   caption=\lstname: pseudo-code for mark-sweep,
                   label=fig:ms:pseudo]
                   
work.addAll(root-set)
for (obj = work.pop(); obj != null;obj = work.pop() ) {
  for (ref : obj.references()) {
    if (ref != null && ref.get().testAndMark()) {
      work.push(ref.get())
    }
  }
}
\end{lstlisting}


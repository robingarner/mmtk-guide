%%This is a very basic article template.
%%There is just one section and two subsections.
\documentclass[11pt]{book}

\usepackage{rotating}

\usepackage{fourier}
\usepackage[T1]{fontenc}

\usepackage{xspace}
\usepackage[hyperindex=true, bookmarks=true,
            pdftitle={MMTk},
            pdfauthor={Robin Garner},
            colorlinks=false,
            pdfborder=0,
            pagebackref=true,
            citecolor=blue,
            plainpages=false,
            pdfpagelabels,
            hyperfootnotes=false]{hyperref}
\renewcommand*{\backref}[1]{}  
   \renewcommand*{\backrefalt}[4]{
      \ifcase #1 
         (not cited)
      \or
         (cited on page #2)
      \else
         (cited on pages #2)
      \fi} 

\usepackage{multirow}
\usepackage{listings}
\lstloadlanguages{java}
\lstset{basicstyle=\footnotesize,language=Java,numbers=left,captionpos=b}

% Includes the \citet and \citep macros
\usepackage[square]{natbib}

\usepackage[obeyDraft,color=todocolor, colorinlistoftodos]{todonotes}

\usepackage{tikzuml/tikz-uml}


\newcommand{\eg}{e.g.\@\xspace} 
\newcommand{\ie}{i.e.\@\xspace}
\newcommand{\etc}{etc.\@\xspace}
\newcommand{\naive}{na\"\i ve\xspace}
\newcommand{\Naive}{Na\"\i ve\xspace}

\newcommand{\java}{Java\xspace} 
\newcommand{\mmtk}{MMTk\xspace}
\newcommand{\jikes}{Jikes RVM\xspace} 
\newcommand{\jikesrvm}{\jikes}


\begin{document}

\part{Introduction}

% Intro
\chapter{Introduction}

This document is intended to be a comprehensive reference manual for MMTk, the Memory Management Toolkit.

The intended audience is anyone intending to create or modify an MMTk collector, and developers of
language runtimes/virtual machines who want to integrate MMTk into their system.
It attempts as far as possible not to pre-suppose any specialist knowledge of Memory Management, Java
Virtual Machines or Jikes RVM.
Conversely it does not attempt to be a comprehensive introduction to any of these subjects, and
readers are encouraged to read various other reference materials which do this task much more successfully.
Instead, I attempt to provide enough background information to understand the MMTk code in question,
providing a description of MMTk's algorithm or implementation and referring the reader to a more
comprehensive reference source for alternative approaches or broader descriptions of tradeoffs.

% About
%
% Intro to GC
\section{Automatic Memory Management}

Programming languages that support dynamic data structures require mechanisms for allocating memory,
and recovering it for re-allocation when it is no longer used.  Languages such as C require the programmer
to track memory usage and to explicitly free memory when it is no longer in use.  Languages such as LISP
and Java use automatic methods called \emph{Garbage Collection} to reclaim unused memory.  These languages
also support features such as \emph{finalizers} and \emph{reference types} that require support from the
garbage collector.

Freeing the programmer from the responsibility of freeing memory leads to a significantly different programming
style to manually managed languages. This leads to more frequent memory allocation, often of objects which
are very short lived.  The efficiency of memory allocation becomes a much more significant performance 
issue for languages with automatic memory management than it is in other languages.  This affects the way
MMTk is written, because saving a couple of instructions (particularly memory references) in an allocator
can have a significant performance effect.


\subsection{Allocation}

Allocation of memory is one of the two major functions of a memory manager.  There are
two basic methods: \emph{bump-pointer} or \emph{monotonic} allocation, where objects
are allocated from a contiguous region of memory by ``bumping a pointer''; and 
\emph{feee-list} allocation, where unused regions of memory are maintained in
a list.

For more information, the survey paper of \citet{WJNB:95} is a good starting place.
% Free List / Bump Pointer
% Unsync/sync

\subsection{Garbage Collection}

% Tracing
% - Copying (evacuating)
% - In-place
% Reference Counting

\subsection{Language Features}
%
% Intro to MMTk
\section{MMTk}

%
% - Key concepts (policy, plan, local/global)
% 



% High level outline of an MMTk plan.  
% - How allocation works
% - How collection works
% - Design criteria and constraints
%   - Thread-local versus global
%   - Inlining
\chapter{Garbage Collection in MMTk}


%
% - Mark-sweep garbage collection
% - The MMTk algorithms
% - The components of the MS plan
%
\chapter{Mark-Sweep in MMTk}

\label{chap:mark-sweep}

\mmtk's implementation of mark-sweep is a highly optimized state-of-the-art
collector, and until recently its generational counterpart, GenMS, was the
default production collector.  The mark-sweep policy that underlies the
plan is used in every \mmtk\ collector since it is used for the non-moving and
code spaces common to all plans.

The basic idea is shared with McCarthey's 1960 collector~\citep{McCarthy:60}:
that the collector performs a transitive closure over the object graph marking 
live objects, and that all objects not marked are considered dead
at the end of the collection cycle.  We traditionally refer to marked objects as
being coloured black, and unmarked objects as white.
\begin{figure}
\begin{lstlisting}
/* 
 * Initially, all objects in the heap are coloured white.
 * white, gray and black are disjoint sets of objects
 */              
for (ref : root-set)
  if (ref != zero)
    gray.add(ref.load())

while (!gray.isEmpty())
  obj = gray.get();
  for (ref : obj.nonNullReferences())
    child = ref.load()
    if (child.isWhite())
      gray.add(child)
  black.add(obj)
/*
 * Now all objects are either black or white, white ones
 * are unreachable and can be freed.
 */
for (obj : white)
  obj.free()
\end{lstlisting}
\caption{Mark Sweep: pseudo-code description}
\label{fig:ms:pseudo}
\end{figure}
The pseudo-code in Figure~\ref{fig:ms:pseudo} illustrates the algorithm in terms
of the traditional tri-colour abstraction from \cite{DLM+:76}.  

As with all \mmtk\ plans, collection is performed in parallel, with a number of
collector threads running in parallel to mark the heap.

The abstract description in Figure~\ref{fig:ms:pseudo} requires that each object
can be assigned to one of three sets: white, black and gray.  Traditionally this
is implemented using a boolean flag per object (either stored in the object or
outside the heap).  If the flag is false, the object is white.  If the flag is
true, then either the object is gray or black.  The set of gray objects is
usually kept as a side data structure such as a stack or a
queue, since the purpose of the 'grey' state is as a list of work
still to be done\footnote{There are other more arcane implementations that can
thread the grey set through the heap.  This is generally only of interest in 
space-constrained LISP-like languages where most objects are only 2 words long.}.

\mmtk\ supports two implementations of the mark state, described in
Section~\ref{sec:ms:state}.  The mark-state is either stored in the header
of each object, or in a region reserved at the start of every 4MB chunk of
virtual address space.

One of the shortcomings of pure mark-sweep collectors is that they must traverse
the heap twice in each collection cycle, once to mark the reachable objects
(the mark phase), and then a second time to build the free list from the white
objects (the sweep phase).  
Black objects can also be marked white during the sweep phase, 
to initialize them ready for the next GC.
The sweep phase is particularly expensive because it requires visiting the
entire heap, not just the (hopefully much smaller) fraction that is in use.
\mmtk\ uses \emph{lazy sweeping} \citep{Hughes:82, Boehm:00} to ameliorate the
cost of sweeping.  The \mmtk\ implementation is described in detail below in
Section~\ref{sec:ms:lazy}.

For allocation, the mark-sweep policy uses a \emph{segregated free list}
allocator, described in detail in Section~\ref{sec:alloc:free-list}.  The
essential features relevant here are that a) the heap is divided into
\emph{cells} of ~20 pre-defined \emph{size classes}.  If there are no free cells
of a given size class, a \emph{block} of 4-64KB is allocated and divided into
cells of the given size class.  Within a block, all cells are therefore of the
same size class, so as long as we know the size class of each
block, it is possible to locate each cell (and therefore each object) in the block.

The collector maintains a list per size-class of all blocks containing live
objects.  Each mutator maintains a list per size-class of the blocks it has
allocated into, and a pointer to the first free cell of each size class.

\begin{diagram}

\caption{Blocks and cells in the segregated free list}
\end{diagram}

\section{Lazy Sweeping}
\label{sec:ms:lazy}



\section{Mark-state implementation}
\label{sec:ms:state}




\begin{lstlisting}[name=Mark Phase, 
                   caption=\lstname: pseudo-code for mark-sweep,
                   label=fig:ms:pseudo]
                   
work.addAll(root-set)
for (obj = work.pop(); obj != null;obj = work.pop() ) {
  for (ref : obj.references()) {
    if (ref != null && ref.get().testAndMark()) {
      work.push(ref.get())
    }
  }
}
\end{lstlisting}



% Generational Collection
\input{generational}

% Concurrency
\input{concurrent}

% Complications, Language Features
\input{features}

\part{Mechanisms}

\chapter{The {\ttfamily Plan} classes}

\label{chap:plan}

% Plans.  Stop-the-world. Generational. Reference Counting. Concurrent.

% Policies: Spaces
\chapter{Policies}

\section{Large Object Space}
\label{sec:policy:LargeObject}
% Virtual memory management

% Queues

% VM Interface(s)

\part{Implementations}

% The MMTk Harness

% Jikes RVM

\chapter*{References}

\bibliographystyle{plainnat}
\bibliography{mmtk-guide}




\end{document}
